%%
%% This is file `sample-sigplan.tex',
%% generated with the docstrip utility.
%%
%% The original source files were:
%%
%% samples.dtx  (with options: `sigplan')
%% 
%% IMPORTANT NOTICE:
%% 
%% For the copyright see the source file.
%% 
%% Any modified versions of this file must be renamed
%% with new filenames distinct from sample-sigplan.tex.
%% 
%% For distribution of the original source see the terms
%% for copying and modification in the file samples.dtx.
%% 
%% This generated file may be distributed as long as the
%% original source files, as listed above, are part of the
%% same distribution. (The sources need not necessarily be
%% in the same archive or directory.)
%%
%% Commands for TeXCount
%TC:macro \cite [option:text,text]
%TC:macro \citep [option:text,text]
%TC:macro \citet [option:text,text]
%TC:envir table 0 1
%TC:envir table* 0 1
%TC:envir tabular [ignore] word
%TC:envir displaymath 0 word
%TC:envir math 0 word
%TC:envir comment 0 0
%%
%%
%% The first command in your LaTeX source must be the \documentclass command.

\documentclass[sigplan,screen]{acmart}
\usepackage{listings}
%% NOTE that a single column version is required for 
%% submission and peer review. This can be done by changing
%% the \doucmentclass[...]{acmart} in this template to 
%% \documentclass[manuscript,screen,review]{acmart}
%% 
%% To ensure 100% compatibility, please check the white list of
%% approved LaTeX packages to be used with the Master Article Template at
%% https://www.acm.org/publications/taps/whitelist-of-latex-packages 
%% before creating your document. The white list page provides 
%% information on how to submit additional LaTeX packages for 
%% review and adoption.
%% Fonts used in the template cannot be substituted; margin 
%% adjustments are not allowed.
%%
%% \BibTeX command to typeset BibTeX logo in the docs
\AtBeginDocument{%
  \providecommand\BibTeX{{%
    \normalfont B\kern-0.5em{\scshape i\kern-0.25em b}\kern-0.8em\TeX}}}

%% Rights management information.  This information is sent to you
%% when you complete the rights form.  These commands have SAMPLE
%% values in them; it is your responsibility as an author to replace
%% the commands and values with those provided to you when you
%% complete the rights form.

%
%  Uncomment \acmBooktitle if th title of the proceedings is different
%  from ``Proceedings of ...''!
%
%\acmBooktitle{Woodstock '18: ACM Symposium on Neural Gaze Detection,
%  June 03--05, 2018, Woodstock, NY} 
% \acmPrice{15.00}
% \acmISBN{978-1-4503-XXXX-X/18/06}


%%
%% Submission ID.
%% Use this when submitting an article to a sponsored event. You'll
%% receive a unique submission ID from the organizers
%% of the event, and this ID should be used as the parameter to this command.
%%\acmSubmissionID{123-A56-BU3}

%%
%% For managing citations, it is recommended to use bibliography
%% files in BibTeX format.
%%
%% You can then either use BibTeX with the ACM-Reference-Format style,
%% or BibLaTeX with the acmnumeric or acmauthoryear sytles, that include
%% support for advanced citation of software artefact from the
%% biblatex-software package, also separately available on CTAN.
%%
%% Look at the sample-*-biblatex.tex files for templates showcasing
%% the biblatex styles.
%%

%%
%% The majority of ACM publications use numbered citations and
%% references.  The command \citestyle{authoryear} switches to the
%% "author year" style.
%%
%% If you are preparing content for an event
%% sponsored by ACM SIGGRAPH, you must use the "author year" style of
%% citations and references.
%% Uncommenting
%% the next command will enable that style.

%%
%% end of the preamble, start of the body of thef document source.
\begin{document}

%%
%% The "title" command has an optional parameter,
%% allowing the author to define a "short title" to be used in page headers.
\title{Enhancing Neurologic Decoding via Structured Knowledge Bases}

%%
%% The "author" command and its associated commands are used to define
%% the authors and their affiliations.
%% Of note is the shared affiliation of the first two authors, and the
%% "authornote" and "authornotemark" commands
%% used to denote shared contribution to the research.
% \author{Rodney Reichert}
% \authornote{}
% \email{rodneyreichert2024@u.northwestern.edu}
% \orcid{}
\author{Kyle Williams}
% \authornotemark[1]
\email{kylewilliams2023@u.northwestern.edu  }
\affiliation{%
  \institution{Northwestern University}
  % \streetaddress{P.O. Box 1212}
  \city{Evanston}
  \state{Illinois}
  \country{USA}
  % \postcode{43017-6221}
}

\author{Rodney Reichert}
% \authornotemark[1]
\email{  rodneyreichert2024@u.northwestern.edu  }
\affiliation{%
  \institution{Northwestern University}
  % \streetaddress{P.O. Box 1212}
  \city{Evanston}
  \state{Illinois}
  \country{USA}
  % \postcode{43017-6221}
}
\author{Yemi Lani}
% \authornotemark[1]
\email{yemikelani2022@u.northwestern.edu}
\affiliation{%
  \institution{Northwestern University}
  \streetaddress{P.O. Box 1212}
  \city{Evanston}
  \state{Illinois}
  \country{USA}
  % \postcode{43017-6221}
}

\author{Arya Bulusu}
% \authornotemark[1]
\email{aryabulusu2024@u.northwestern.edu}
\affiliation{%
  \institution{Northwestern University}
  % \streetaddress{P.O. Box 1212}
  \city{Evanston}
  \state{Illinois}
  \country{USA}
  % \postcode{43017-6221}
}

%%
%% By default, the full list of authors will be used in the page
%% headers. Often, this list is too long, and will overlap
%% other information printed in the page headers. This command allows
%% the author to define a more concise list
%% of authors' names for this purpose.
\renewcommand{\shortauthors}{TEMP}
\settopmatter{printacmref=false}

%%
%% The abstract is a short summary of the work to be presented in the
%% article.
\begin{abstract}
Large Language Models (LLMs) demonstrate proficiency in various general tasks and can acquire human level knowledge during training. However, there is a lack of inherent mechanisms to ensure the accuracy of an LLMs output. Contrany to LLms, knowledge bases serve as valuable resources containing vetted relationships, but incorporating these relationships into the output of an LLM presents a significant challenge. Transformer architectures rely on extensive training corpora, thus most knowledge bases contain insufficient data to train a model from scratch or constitute a meaningful portion of the model's training data. Moreover, there is no guarantee that the model accurately memorizes the facts from the knowledge base. Moreover, determining the most effective representation of knowledge from the base in natural language is a complex question in itself. Therefore, we propose the utilization of Neurologic Decoding in conjunction with a knowledge base as a means to enhance factual information generation.
\end{abstract}

%%
%% The code below is generated by the tool at http://dl.acm.org/ccs.cfm.
%% Please copy and paste the code instead of the example below.
%%


%%
%% Keywords. The author(s) should pick words that accurately describe
%% the work being presented. Separate the keywords with commas.
\keywords{TODO:}

%% A "teaser" image appears between the author and affiliation
%% information and the body of the document, and typically spans the
%% page.
% \begin{teaserfigure}
%   \includegraphics[width=\textwidth]{sampleteaser}
%   \caption{Seattle Mariners at Spring Training, 2010.}
%   \Description{Enjoying the baseball game from the third-base
%   seats. Ichiro Suzuki preparing to bat.}
%   \label{fig:teaser}
% \end{teaserfigure}


%%
%% This command processes the author and affiliation and title
%% information and builds the first part of the formatted document.
\maketitle

\section{Introduction}

Talk about LLMs solving certian tasks (i.e machine translation)

Talk about LLMs not doing to well on other tasks such as common sense reasoning or mulit hop reasoning. 

\section{Previous Work}

Talk about Neurologic

Talk about seeding the input of LLMs with triples from KBs

\section{Model}

\section{Dataset}

\section{Experiments}

\section{Results}

\section{Discussion}

\section{Conclusion}





\end{document}


\end{document}
\endinput
%%
%% End of file `sample-sigplan.tex'.
